\documentclass[a4paper,12pt]{article}
\usepackage[margin=1in]{geometry}
\usepackage{listings}
\begin{document}
\lstset{frame=single,tabsize=4,basicstyle=\ttfamily}

{\centering \bf \Large
Lecture 1 Notes \\[\baselineskip]
}

\subsection*{Introduction}

CSE 201 is one of the first CSE courses that students at CSUSB enroll in. Many new students are unaware of what Computer Science demands from it's potential scholars. It is first important to know that we are not studying the computer; the computer is simply a tool for us to apply our theories. What we are studying is: not the computer, but computation itself, or (in other words) algorithms. An algorithm is a series of instructions for performing a particular task. 

Creating and applying algorithms involves significant application of logic and mathematics. Inside a computer, an algorithm can physically exist in the form of a computer program. The process of creating a computer program is called computer programming.

CSE 201 serves as an introduction to computer programming. We will use a programming language known as C++.

\subsection*{Your first program}

The first program that you usually create in any programming language is known as the ``hello world'' program. The only function of this program is to display a message. In C++ that program can be expressed in the following form:

\begin{lstlisting}[caption=hello.cpp]
#include <iostream>
using namespace std;

int main()
{
	cout << "hello world" << endl;
	return 0;
}
\end{lstlisting}

This is what we call the source code of a computer program. Source code is the human readable version of a computer program's algorithm. We must convert source code into a computer program before we can execute it (a process known as compiling), this is done by using a program known as a compiler. 

Compilers as well as other programmer tools can be executed through the operating system's command-line interface. 

If you are using any UNIX-like operating system (including Linux and Mac OS X), you can compile a source file named hello.cpp by typing this command into your command-line interface:
\begin{lstlisting}
c++ hello.cpp
\end{lstlisting}
A program named ``a.out'' will be produced.  You may execute the a.out program with the following command:
\begin{lstlisting}
./a.out
\end{lstlisting}

Take note that the compiler and the hello world program both execute within an environment known as the command-line interface. This environment has text-only graphics; devoid of any buttons, scrollbars, or images. This is a programmer-friendly environment that is not only easier to create programs in, but also extends more power to the user. 

\subsection*{Data types}

A computer program is usually created for processing data. Processing data requires being able to distinguish between different types of data. The C++ programming language supports many elementary data types, four of them that you should know are in the following table: \\

\begin{tabular}{|l|l|l|}
\hline
{\bf Data type} & {\bf Size} & {\bf Range} \\
\hline
bool & 1 byte & true or false \\
\hline
char & 1 byte & -128 to 127 \\
\hline
int & 4 bytes & -2147483648 to 2147483647 \\
\hline
double & 8 bytes & $\pm1.7 \times 10^{308}$ \\
\hline
\end{tabular} \\

bool is short for boolean. Data of type bool can have one of two values: true or false. Some expressions in C++ often result in a boolean value, this is typically the case when you are mathematically comparing two values.

char is short for character. It represents almost any key that you can press from your keyboard. A character can have a value such as: `a', `3', `\%', `.', ` ', `J'. All text that you see on your computer screen is stored in the computer's memory as sequences of characters. While C++ allows you to express characters as their symbol in single-quotes, each character actually represents an integer within the range of 1 byte.

int stands for integer. It represents a negative or positive whole number. Integers are useful for basic arithmetic where fractional values are not required. 

double stands for double-precision floating-point. You do not need to know what that means exactly, just know that double is used for representing fractional numbers. If you are performing any scientific or financial calculations, you should use this data type as opposed to an integer. 

There are many other basic data types in C++, some of them are variations of the int and double data types. In this course, we will stick to these four basic data types.

\subsection*{Variables}

In addition to distinguishing between different types of data, computer programs should have a way to store data. C++, as well as many other programming languages, allow you to do this by declaring variables.

In C++, all variables have a data type, a name, and a value. To declare a variable, you must specify at least the data type and the name of the variable. For instance:

\begin{lstlisting}[caption=A series of variable declarations in C++]
int x;
double amount;
bool cointoss;
int weight = 150;
double balance = 5071.12;
char q = 'A';
bool verified = true;
\end{lstlisting}


\end{document}
