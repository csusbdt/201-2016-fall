\documentclass[a4paper,12pt]{article}
\usepackage[margin=1in]{geometry}
\usepackage{listings}
\begin{document}
\lstset{frame=single,tabsize=4,basicstyle=\ttfamily}

{\centering \bf \Large
Lecture 1 Notes \\[\baselineskip]
}

\subsection*{Your first program}

The first program that you usually create in any programming language is known as the ``hello world'' program. The only function of this program is to display a message. In C++ that program can be expressed in the following form:

\begin{lstlisting}[caption=hello.cpp]
#include <iostream>
using namespace std;

int main()
{
	cout << "hello world" << endl;
	return 0;
}
\end{lstlisting}

This is what we call the source code of a computer program. Source code is the human readable version of a computer program's algorithm. We must convert source code into a computer program before we can execute it (a process known as compiling), this is done by using a program known as a compiler. 

Compilers as well as other programmer tools can be executed through the operating system's command-line interface. 

Assuming that you are using Linux, if you saved hello.cpp in your Documents folder, then you can compile it by typing these commands into your terminal emulator:
\begin{lstlisting}
cd ~/Documents
c++ hello.cpp
\end{lstlisting}
A program named ``a.out'' will be produced.  You may execute the a.out program with the following command:
\begin{lstlisting}
./a.out
\end{lstlisting}

Take note that the compiler and the hello world program both execute within an environment known as the command-line interface. This environment has text-only graphics; devoid of any buttons, scrollbars, or images. This is a programmer-friendly environment that is not only easier to create programs in, but also extends more power to the user. 

\subsection*{Data types}

A computer program is usually created for processing data. Processing data requires being able to distinguish between different types of data. The C++ programming language supports many elementary data types, four of them that you should know are in the following table: \\

\begin{tabular}{|l|l|l|}
\hline
{\bf Data type} & {\bf Size} & {\bf Range} \\
\hline
bool & 1 byte & true or false \\
\hline
char & 1 byte & -128 to 127 \\
\hline
int & 4 bytes & -2147483648 to 2147483647 \\
\hline
double & 8 bytes & $\pm1.7 \times 10^{308}$ \\
\hline
\end{tabular} \\

bool is short for boolean. Data of type bool can have one of two values: true or false. Some expressions in C++ often result in a boolean value, this is typically the case when you are mathematically comparing two values.

char is short for character. It represents almost any key that you can press from your keyboard. A character can have a value such as: `a', `3', `\%', `.', ` ', `J'. All text that you see on your computer screen is stored in the computer's memory as sequences of characters. While C++ allows you to express characters as their symbol in single-quotes, each character actually represents an integer within the range of 1 byte.

int stands for integer. It represents a negative or positive whole number. Integers are useful for basic arithmetic where fractional values are not required. 

double stands for double-precision floating-point. You do not need to know what that means exactly, just know that double is used for representing fractional numbers. If you are performing any scientific or financial calculations, you should use this data type as opposed to an integer. 

There are many other basic data types in C++, some of them are variations of the int and double data types. In this course, we will stick to these four basic data types.

\subsection*{Variables}

In addition to distinguishing between different types of data, computer programs should have a way to store data. C++, as well as many other programming languages, allow you to do this by declaring variables.

In C++, all variables have a data type, a name, and a value. To declare a variable, you must specify at least the data type and the name of the variable. For instance:

\begin{lstlisting}[caption=A series of variable declarations in C++]
int x;
double amount;
bool cointoss;
int weight = 150;
double balance = 5071.12;
char q = 'A';
bool verified = true;
\end{lstlisting}

As you can see, variables can optionally have a value assigned upon declaration. If no value is assigned, the default value is used: \\

\begin{tabular}{|l|l|}
\hline
{\bf Data type} & {\bf Default value} \\
\hline
bool & false \\
\hline
char & '\textbackslash 0' \\
\hline
int & 0 \\
\hline
double & 0.0 \\
\hline
\end{tabular} \\

After a variable has been declared, it's stored value can be fetched by simply referring to the name of the variable. 

\subsection*{Expressions and Statements}

Formally, every C++ program is a sequence of statements. Each statement performs an action. There are two types of C++ statements, compiler statements and preprocessor statements. 

Preprocessor statements begin with a `\#' symbol at the start of a line. These type of statements are used for modifying or generating compiler statements. For example, the `\#include' preprocessor statment is used for inserting source code from an external file into the current source code. 

Compiler statements describe what the program will do. Compiler statements are delimited by a semicolon, and you may have multiple of these statements on a single line.

Some statements result in a value which can be assigned to a variable, in which case the statement is also an expression. All expressions are statements, but not all statements are expressions. 

This is a statement: 

\begin{lstlisting}
cout << "hello.";
\end{lstlisting}

The action this statement performs is the displaying of the character sequence ``hello.''. This statement does not produce a value.

This is an expression:

\begin{lstlisting}
a + 4;
\end{lstlisting}

This expression calculates the value of variable a added to the value 4. This produces a value which can be assigned to a variable, displayed to the screen, or used in another statement. 

\subsection*{Analysis of the Hello World program}

Let us return to your first program, the hello world program. Here is hello.cpp again with line numbers on the left:

\lstset{numbers=left}
\begin{lstlisting}
#include <iostream>
using namespace std;
int main()
{
	cout << "hello world" << endl;
	return 0;
}
\end{lstlisting}

Line 1 is a preprocessor statement, iostream is the name of a file included in every C++ distribution. iostream contains statements that allows us to receive and display messages in our terminal window. One variable that iostream exports is the ``cout'' variable. cout stands for character out. All data that you write into cout gets displayed in your terminal window, or more generally what we call ``standard output''. 

Line 2 is related to a concept in C++ known as namespaces. In C++, when we declare variables or functions we can choose to place them inside a namespace. Namespaces can exist inside other namespaces too. The perceived benefit of namespaces is to prevent source code from different programmers from conflicting with each other if their code happens to use the same variable names. To access a namespaced variable, requires that you type the name of all the containing namespaces as well as the variable name. The built-in source code bundled with every C++ distribution exists under the namespace called ``std''. All variables that do not exist inside a declared namespace are said to exist in the global namespace. 

The ``using namespace std'' statement means that we wish to take all variables from inside the std namespace and move them into the global namespace. Without this statement, we would have to type ``std::cout'' every time we wish to use cout. We would also have to type ``std::'' before using any other variable from the C++ library. This command allows us to use cout without typing the preceeding ``std::''.

Line 3 is the declaration of a function called main. You will learn more about functions later on in this course, for now you should just know that the main function signifies the starting point of your program's execution. 

Lines 4 and 7 encloses the sequence of statements that gets executed inside the main function.

Line 5 is the statement that displays the character sequence ``hello world''. Recall that a value of type char is enclosed in single-quotes. C++ allows us to create a sequence of characters by using double-quotes. This sequence of characters is what we call a C-string. C-strings have the type ``const char *'' which you will learn more about later. On line 5, we are taking the C-string value ``hello world'' and writing it into our variable ``cout''. The $<<$ operator indicates that we are writing data from the right-hand side of the operator to the variable on the left-hand side of the operator. The $<<$ operator used in this context only works on variables that are of type ``ostream''. ostream is a special data type included in C++. You may write multiple data into an ostream object in one statement by chaining the $<<$ operator. In our example, we write both ``hello world'' and the value of variable endl into cout. 

endl is a special built-in variable that inserts a line-break into any ostream object that it is written to.

Line 6 is the final statement that gets executed while our program is running. It instructs the execution of our main function to stop and give the value zero as the exit status code to our operating system. After running any program in our terminal, you may use the following command to check that program's exit status:

\lstset{numbers=none}
\begin{lstlisting}
echo $?
\end{lstlisting}

The result will be 0 if the program exited normally, any non-zero value indicates an error.

You may change the value in hello.cpp on line 6 to something like ``return 2;''. After recompiling and running your program, use the ``echo \$?'' command to see the exit status.

\end{document}
