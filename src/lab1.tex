\documentclass[a4paper,12pt]{article}
\usepackage[margin=1in]{geometry}
\usepackage{listings}

\setlength{\parindent}{0pt}

\begin{document}

\lstset{frame=single,basicstyle=\ttfamily,tabsize=4}

{\centering
\large \bf
Lab 1
}

Place all your work inside a folder named ``lab1''.

\subsubsection*{Exercise 1}
\begin{lstlisting}[caption=hello.cpp]
#include <iostream>
using namespace std;

int main()
{
	cout << "hello, world" << endl;
	return 0;
}
\end{lstlisting}
Compile and run the above program on your system

\subsubsection*{Exercise 2}
The following program calculates the area of a rectangle:
\begin{lstlisting}[caption=area.cpp]
#include <iostream>
using namespace std;

int main()
{
	double length;
	double width;
	double area;

	cout << "Enter the length: ";
	cin >> length;
	cout << "Enter the width: ";
	cin >> width;
	area = length * width;
	cout << "The area is: " << area << endl;
	return 0;
}
\end{lstlisting}
Modify this program to calculate the volume of a rectangular prism after 3 side lengths are input from the user.

\subsubsection*{Exercise 3}
This program displays a greeting after asking for your name.
\begin{lstlisting}[caption=name.cpp]
#include <iostream>
using namespace std;

int main()
{
	char firstname[32];

	cout << "What is your first name? ";
	cin >> firstname;
	
	cout << "Hello, " << firstname << endl;
	return 0;
}
\end{lstlisting}
Modify the above program to make it ask the user for their first and last name. Then make it display ``Hello, '' followed by their first and last name.

\end{document}
