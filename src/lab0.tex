\documentclass[a4paper,12pt]{article}
\usepackage[margin=1in]{geometry}

\begin{document}

{\centering \Large \bf
Lab 0 \\[\baselineskip]
}

This lab will prepare you to use a particular programming environment which you will use for the rest of this course. This lab has two parts.

\subsubsection*{Set up free acount with Cloud9}

Cloud9 is an online web-host and shell account with a web-based Integrated Development Environment (IDE). It runs the same underlying operating system as our CSE lab computers but with a different interface. We will have you use this service so that it will be easier for the instructors to access your work. 

Please follow these instructions for this part:
\begin{enumerate}
\item Create a free Cloud9 account at http://c9.io, use your full name as your new username.
\item Create a new ``workspace''. Title the workspace ``cse201''. Have it be a ``Hosted workspace'' with ``Public'' access. Use the C++ template.
\item Inside your workspace, click ``Share'' in the top right hand corner. 
\item Under the ``Invite People'' box, invite each instructor using their Cloud9 usernames found in the syllabus, make sure ``RW'' is highlighted when you invite them. RW means Read-Write access.
\end{enumerate}

From now on, the instructors will automatically check your cse201 workspace on Cloud9 at the end of each week. Each time we grade your lab, we will leave a note in your workspace.

\subsubsection*{Learn the Command-Line at codecademy}

Codecademy is a source for interactive programming tutorials. This will be the only lab where we ask you to complete a tutorial from this website. 

Please follow these instructions for this part:
\begin{enumerate}
\item Create a free account at codecademy. Use your full name as your username. 
\item Complete the command-line tutorial at: https://www.codecademy.com/learn/learn-the-command-line
\item Go to your account settings, and set the option to ``Everyone'' on ``Who can view my profile''. In case you've forgotten, you should also set your username here to your fullname.
\end{enumerate}

At the end of this week only, we will check your profile to see that you have completed the command-line tutorial.

\end{document}
